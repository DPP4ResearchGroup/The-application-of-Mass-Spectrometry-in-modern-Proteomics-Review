The resolution of the mass separator depends on the m/z ratio
Mass error rate