\label{fig:ionisation}
Principals of different ionization sources in mass spectrometry. A) Secondary ion mass spectrometry (SIMS) uses high energy primary ion beam bombard the section surface and transfer energy absorbed by the analytes to form secondary ions. B) Matrix assisted laser desorption ionisation (MALDI) is arguable the most popular ionisation technique in biomedical research. MALDI utilizes UV range laser photons as high energy source and high energy is absorbed by matrix layer and consequently cause analytes to ionize and fragment in a formatted fashion. C) Desorption electrospray ionization (DESI) maintains a concentrated charged versus surface ratio results the fragmentation and ionisation of analytes molecules. D) Laser ablation electrospray ionisation (LA-ESI) adopts mid IR range primary laser beam to generate secondary analyte ion source, which commonly exit in multi-charge ions forms. E) Laser ablation inductively coupled plasma (LA-ICP). F) Nano-desorption electrospray ionization (nano-DESI) G) Liquid extraction surface anaysis (LESA) H) Low temperature plasma adopts relative soft plasma beam to ionise delicate surface metabolites. I) Primary MALDI source coupled with secondary MALDI to generate secondary analyte ions from the ablation plume. 